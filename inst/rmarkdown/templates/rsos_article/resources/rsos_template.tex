%% Author_tex.tex
%% V1.0
%% 2012/13/12
%% developed by Techset
%%
%% This file describes the coding for rsproca.cls

\documentclass[]{rsos}%%%%where rsos is the template name

%%%% *** Do not adjust lengths that control margins, column widths, etc. ***

%%%%%%%%%%% Defining Enunciations  %%%%%%%%%%%
\newtheorem{theorem}{\bf Theorem}[section]
\newtheorem{condition}{\bf Condition}[section]
\newtheorem{corollary}{\bf Corollary}[section]
%%%%%%%%%%%%%%%%%%%%%%%%%%%%%%%%%%%%%%%%%%%%%%%


\begin{document}

%%%% Article title to be placed here
\title{Insert the article title here}

\author{%%%% Author details
X. X. First author$^{1}$, X. Second author$^{2}$ and X. Third author$^{3}$}

%%%%%%%%% Insert author address here
\address{$^{1}$First author address\\
$^{2}$Second author address\\
$^{3}$Third author address}

%%%% Subject entries to be placed here %%%%
\subject{xxxxx, xxxxx, xxxx}

%%%% Keyword entries to be placed here %%%%
\keywords{xxxx, xxxx, xxxx}

%%%% Insert corresponding author and its email address}
\corres{Insert corresponding author name\\
\email{xxx@xxxx.xx.xx}}

%%%% Abstract text to be placed here %%%%%%%%%%%%
\begin{abstract}
The abstract text goes here. The abstract text goes here. The abstract text goes here. The abstract text goes here.
The abstract text goes here. The abstract text goes here. The abstract text goes here. The abstract text goes here.
\end{abstract}
%%%%%%%%%%%%%%%%%%%%%%%%%%%

%%%%%%%%%% Insert the texts which can accomdate on firstpage in the tag "fmtext" %%%%%

\begin{fmtext}
\section{Insert A head here}
%%%% Insert A head here

This demo file is intended to serve as a ``starter file''
for rsproca journal papers produced under \LaTeX\ using
rsproca.cls v1.5e.

\subsection{Insert B head here}
%%%% Insert B head here
Subsection text here.

\subsubsection{Insert C head here}
%%%% Insert C head here
Subsubsection text here.

\section{Equations}

Sample equations.

%%% Numbered equation
\begin{align}\label{1.1}
\begin{split}
\frac{\partial u(t,x)}{\partial t} &= Au(t,x) \left(1-\frac{u(t,x)}{K}\right)-B\frac{u(t-\tau,x) w(t,x)}{1+Eu(t-\tau,x)},\\
\frac{\partial w(t,x)}{\partial t} &=\delta \frac{\partial^2w(t,x)}{\partial x^2}-Cw(t,x)+D\frac{u(t-\tau,x)w(t,x)}{1+Eu(t-\tau,x)},
\end{split}
\end{align}

\begin{align}\label{1.2}
\begin{split}
\frac{dU}{dt} &=\alpha U(t)(\gamma -U(t))-\frac{U(t-\tau)W(t)}{1+U(t-\tau)},\\
\frac{dW}{dt} &=-W(t)+\beta\frac{U(t-\tau)W(t)}{1+U(t-\tau)}.
\end{split}
\end{align}

%%%% Unnumbered equation
\begin{eqnarray}
\frac{\partial(F_1,F_2)}{\partial(c,\omega)}_{(c_0,\omega_0)} = \left|
\begin{array}{ll}
\frac{\partial F_1}{\partial c} &\frac{\partial F_1}{\partial \omega} \\\noalign{\vskip3pt}
\frac{\partial F_2}{\partial c}&\frac{\partial F_2}{\partial \omega}
\end{array}\right|_{(c_0,\omega_0)}\notag\\
=-4c_0q\omega_0 -4c_0\omega_0p^2 =-4c_0\omega_0(q+p^2)>0.
\end{eqnarray}
\end{fmtext}

%%%%%%%%%%%%%%% End of first page %%%%%%%%%%%%%%%%%%%%%

\maketitle

\section{Enunciations}

\begin{theorem}\label{T0.1}
Assume that $\alpha>0, \gamma>1, \beta>\frac{\gamma+1}{\gamma-1}$.
Then there exists a small $\tau_1>0$, such that for $\tau\in
[0,\tau_1)$, if $c$ crosses $c(\tau)$ from the direction of
to  a small amplitude periodic traveling wave solution of
(2.1), and the period of $(\check{u}^p(s),\check{w}^p(s))$ is
\[
\check{T}(c)=c\cdot \left[\frac{2\pi}{\omega(\tau)}+O(c-c(\tau))\right].
\]
\end{theorem}


\begin{condition}\label{C2.2}
From (0.8) and (2.10), it holds
$\frac{d\omega}{d\tau}<0,\frac{dc}{d\tau}<0$ for $\tau\in
[0,\tau_1)$. This fact yields that the system (2.1) with delay
$\tau>0$ has the periodic traveling waves for smaller wave speed $c$
than that the system (2.1) with $\tau=0$ does. That is, the
delay perturbation stimulates an early occurrence of the traveling waves.
\end{condition}


\section{Figures \& Tables}

The output for figure is:

\begin{figure}[!h]
%\centering\includegraphics[width=2.5in]{xxxxxx.eps}
%%% where xxxxxx name represents "figurename.eps"
\caption{Insert figure caption here}
\label{fig_sim}
\end{figure}

\vspace*{-10pt}

\noindent The output for table is:

\begin{table}[!h]
\caption{An Example of a Table}%%%Table caption goes here
\label{table_example}
\begin{tabular}{llll}%%%The number of columns has to be defined here
\hline
date &Dutch policy &date &European policy \\
\hline
1988 &Memorandum Prevention &1985 &European Directive (85/339) \\
1991--1997 &{\bf Packaging Covenant I} & & \\
1994 &Law Environmental Management &1994 &European Directive (94/62) \\
1997 &Agreement Packaging and Packaging Waste & & \\
1998--2002 &{\bf Packaging Covenant II} & & \\
2003--2005 &{\bf Packaging Covenant III} & & \\
2006--2007 &{\bf Decree on Packaging and paper} & & \\\hline
\end{tabular}
\end{table}%%%End of the table

\section{Conclusion}
The conclusion text goes here.

\vskip1pc

\ethics{Insert ethics text here.}

\dataccess{Insert data access text here.}

\aucontribute{Insert author contribute text here.}

\competing{Insert competing text here.}

\funding{Insert funding text here.}

\ack{Insert acknowledgment text here.}

\disclaimer{Insert disclaimer text here.}



%%%%%%%%%% Insert bibliography here %%%%%%%%%%%%%%

\begin{thebibliography}{9}

\bibitem{1} Allwood JM, Cullen JM. 2011 \textit{Sustainable materials:  with both eyes open}.
Cambridge, UK: UIT Cambridge. See \href{http://www.withbotheyesopen.com}{http://www.withbotheyesopen.com}.

\bibitem{2}  MacKay DJC. 2008  \textit{Sustainable energy:  without the hot air}.
 Cambridge, UK: UIT Cambridge. See \href{http://www.withouthotair.com}{http://www.withouthotair.com}.

\bibitem{3} Gallman PG. 2011  \textit{Green alternatives and national energy strategy: the facts
 behind the headlines}.  Baltimore,\ MD: Johns Hopkins University Press.

\bibitem{4} MacKay DJC. 2013.  Solar energy in the context of energy use, energy transportation, and
 energy storage. \textit{Proc. R. Soc. A} \textbf{371}.

\end{thebibliography}

\end{document}
